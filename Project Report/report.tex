\documentclass[12pt]{article}
\usepackage[utf8]{inputenc}
\usepackage[a4paper, total={7.5in, 10in}]{geometry}
\usepackage{hyperref}
\usepackage{float}
\usepackage{amsmath}
\usepackage{amssymb}
\usepackage{listings}
\usepackage{spverbatim}
\usepackage{graphicx}
\usepackage{titling}
\usepackage{xcolor}
\include{risc-v.tex}
\renewcommand\maketitlehooka{\null\mbox{}\vfill}
\renewcommand\maketitlehookd{\vfill\null}
\title{\textbf{\Huge Computer Architecure Lab}\\\textbf{\Huge Project Report}\\\vspace{30pt}Nimra Sohail\hspace{95pt}ns06867\\Areeb Adnan Khan\hspace{43pt}ak06865\\Hamad Abdul Razzaq\hspace{22pt}hr06899\\\vspace{30pt}\textbf{R.A: Miss Aimen Najeeb}}
\date{}
\begin{document}
\maketitle
\pagebreak
\section*{\Huge Task 1 - Implementing Sorting Algorithm on RISC-V Single Cycle Processor}
\subsection*{\Large Selection of Algorithm, Assembly Language Code \& Verification}
The selected Sorting Algorithm is the Insertion Sort Algorithm. Its Assembly Language Code can be found in \hyperref[ins_sort]{Module 1.1} in the Appendix. The Module was tested on a RISC-V Simulator and the initial and final arrays can be seen in \hyperref[img1]{Figure 1} and \hyperref[img2]{Figure 2} respectively.
\begin{figure}[H]
    \centering
    \includegraphics[scale = 1]{../images/Unsorted Array.png}
    \label{img1}
    \caption{Initialized (Unsorted) Array}
\end{figure}
\begin{figure}[H]
    \centering
    \includegraphics[scale = 1]{../images/Sorted Array.png}
    \label{img2}
    \caption{Sorted Array}
\end{figure}
\subsection*{\Large Modifications in Single Cycle Processor \& Challenges Addressed}
For executing the assembly language code of insertion sort on our lab-made single cycle processor, Following were the required modifications and the way we addressed it.
\begin{itemize}
    \item \textbf{Inclusion of \texttt{blt, bge, bne, beq} Instructions} - As any Sorting Algorithm requires less than, greater than, equal to and not equal to comparisions, therefore there was a need to include these four instructions. To do so, a seperate module named \hyperref[branch_module]{Branch Module} was created. This module, in addition with the Zero Flag also takes a Positve Flag as an input which indicates whether the Operand 1 is greater than Operand 2 or not. The Positive Flag is outputed from \hyperref[alu]{ALU} by simply looking at the most significant bit of the result. So first, we check the branch signal to see if the instruction is a branch instruction. Then, to differentiate between these four branch instructions, we compare their funct3 values along with the zero and the positive flags and the instruction whose specified conditions are matched is set to high and at the same time, the other instructions are set to low. This module also includes a \texttt{to\_branch} output which is high if branch condition is met and any of the four funct3 values are matched. So, the AND Gate from our previous design is replaced by this branch module.
    \item \textbf{Addition of Load Word and Store Word Instructions} - The Load Word and Store Word functionality was added in the \hyperref[dmem]{Data Memory}. This was required because our insertion sort algorithm is implemented on a word array. So, inorder to differentiate between the Load and Store instructions for Word and Double, the funct3 values of the instructions were checked and based on that the load and store instructions for word and double were integrated.
    \item \textbf{Array Location} - In the \hyperref[ins_sort]{Insertion Sort Code}, we can see that the Base Adress Given was of 200. This base address was changed to 0 while executing on the Single Cycle Processor. In \hyperref[https://venus.kvakil.me]{Venus}, an Online Simulator for RISC-V, the Instruction and Data Memory are the same. So initializing base address to 0 will overwrite the instruction address. That's why we have to go for the base address of 200 over there.
\end{itemize}
\subsection*{\Large Testing and Verification}
For testing purposes, There were 10 outputs added to the \hyperref[dmem]{Data Memory} module. These 10 outpus are for the 10 different indices of the array to be sorted. This is not the part of the Single Cycle Processor, rather they are just hard wired to make the verification easier. Upon Converting the assembly language code into machine code and initializing it in the \hyperref[imem]{Instruction Memory}, the code was executed and the sorted array can be verified from the waveform shown in \hyperref[img3]{Figure 3}.
\begin{figure}[H]
    \centering
    \includegraphics[scale = 0.5]{../images/SCP Wave.png}
    \label{img3}
    \caption{Waveform - Single Cycle Processor}
\end{figure}
\section*{\Huge Task 2 - RISC-V Pipelined Architecture}
\subsection*{\Large Inclusion of 5 Stages and Integration}
To Convert the Single Cycle Processor into Pipeline, 5 Stages were made which were synchornized with the clock. These 5 stages are of IF (Instruction Fetch), ID (Instruction Decode), EX (Execution), MEM (Data Memory), WB (Write Back)  which are divided by the help of \hyperref[if_id]{IF/ID}, \hyperref[id_ex]{ID/EX}, \hyperref[ex_mem]{EX/MEM} and \hyperref[mem_wb]{MEM/WB} stage registers. These stage registers were implemented and then they were integrated in the \hyperref[tlmp]{Top Level Module} to make our processor pipelined.
\subsection*{\Large Testing Instructions in Isolation}
The 3 instructions listed in \hyperref[iso]{Module 1.2} of Appendix were tested separately to verify the working of pipeline. The Following three waveforms shows the different stages of the pipeline during the execution of each of these instructions.
\begin{figure}[H]
    \centering
    \includegraphics[scale = 0.6]{../images/Test 1.PNG}
    \label{img4}
    \caption{Waveform for the instruction \texttt{add x3, x0, x1}}
\end{figure}
\begin{figure}[H]
    \centering
    \includegraphics[scale = 0.5]{../images/Test 2.PNG}
    \label{img5}
    \caption{Waveform for the instruction \texttt{addi x4, x0, 3}}
\end{figure}
\begin{figure}[H]
    \centering
    \includegraphics[scale = 0.51]{../images/Test 3.PNG}
    \label{img6}
    \caption{Waveform for the instruction \texttt{lw x7, 4(x0)}}
\end{figure}
\pagebreak
\section*{\Huge Task 3 - Detection of Hazards}
\subsection*{\Large Handling Data Hazards}
For handling Data Hazards, the \hyperref[forward]{Forwarding Unit} and the \hyperref[hdu]{Hazard Detection Unit} were integrated to the pipelined architecture. The Forwarding Unit is used along with 2 additional MUX for bypassing the value of ALU Result from MEM stage to the EX stage, or the value of either ALUResult or the Data Memory Output from the WB Stage to the EX Stage. The Hazard Detection Unit handles the case where the data is being loaded into a register and that register is used in the instruction following the load instruction. To handle this, we introoduce a stall. This is done by not changing the values in the \hyperref[pc]{Program Counter} and \hyperref[if_id]{IF/ID} stages and setting the control values to zero in \hyperref[id_ex]{ID/EX} Stage to make the instruction analogous to stall. This is done by adding a Write signal to the first two stages which is set low when the values should remain unchanged and the Control of the third stage is set to zero by inluding an 8 by 2 \hyperref[muxc]{MUX} with one set of inputs hardwired to low.
\subsection*{\Large Handling Control Hazards}
The Control Hazard here is handled in the deafult way, that is, by stalling thrice whenever we have to branch. To do this the \texttt{to\_branch} signal coming from the \hyperref[branch_module]{Branch Module} is sent as a Flush signal to the \hyperref[if_id]{IF/ID}, \hyperref[id_ex]{ID/EX} and \hyperref[ex_mem]{EX/MEM} stages. The Flush signal simply makes the Control part of the instructions zero which makes the next 3 instructions equivalent to stall. After that, the branch instruction is fetched and Hence, the branch is made.
\subsection*{\Large Running Insertion Sort on Pipelined Processor}
To check the final functionality of the Pipelined Processor with hazards handled, The Insertion Sort Code was executed on it and the following Figure verifies the sorting code working to perfection on the pipelined architecture.
\begin{figure}[H]
    \centering
    \includegraphics[scale = 0.5]{../images/PipeLine Wave.PNG}
    \label{img6}
    \caption{Sorted Array Waveform - Pipelined Architecture}
\end{figure}
\section*{\Huge Appendix}
\subsection*{\Large Section 1 - Assembly Language Codes}
\subsubsection*{\large Module 1.1 - Insertion Sort}
\label{ins_sort}
\lstinputlisting[language={[RISC-V]Assembler}]{../Assembly Codes/insertion_sort.asm}
\subsubsection*{\large Module 1.2 - Isolation Tests}
\label{iso}
\lstinputlisting[language={[RISC-V]Assembler}]{../Assembly Codes/isolation_tests.asm}
\subsection*{\Large Section 2 - RISC-V Single Cycle Processor Modules}
\subsubsection*{\large Module 2.1 - Adder}
\lstinputlisting[language = Verilog]{../RISC_V_Single_Cycle/Adder.v}
\subsubsection*{\large Module 2.2 - ALU (64-bit)}
\label{alu}
\lstinputlisting[language = Verilog]{../RISC_V_Single_Cycle/ALU_64_bit.v}
\subsubsection*{\large Module 2.3 - ALU Control}
\lstinputlisting[language = Verilog]{../RISC_V_Single_Cycle/ALU_Control.v}
\subsubsection*{\large Module 2.4 - Branch Module}
\label{branch_module}
\lstinputlisting[language = Verilog]{../RISC_V_Single_Cycle/branch_module.v}
\subsubsection*{\large Module 2.5 - Control Unit}
\lstinputlisting[language = Verilog]{../RISC_V_Single_Cycle/Control_Unit.v}
\subsubsection*{\large Module 2.6 - Data Memory}
\label{dmem}
\lstinputlisting[language = Verilog]{../RISC_V_Single_Cycle/Data_Memory.v}
\subsubsection*{\large Module 2.7 - Immediate Data Generator}
\lstinputlisting[language = Verilog]{../RISC_V_Single_Cycle/imm_data_gen.v}
\subsubsection*{\large Module 2.8 - Instruction Decoder}
\lstinputlisting[language = Verilog]{../RISC_V_Single_Cycle/instruction.v}
\subsubsection*{\large Module 2.9 - Instruction Memory}
\label{imem}
\lstinputlisting[language = Verilog]{../RISC_V_Single_Cycle/Instruction_Memory.v}
\subsubsection*{\large Module 2.10 - MUX (64-bit, 2 by 1)}
\lstinputlisting[language = Verilog]{../RISC_V_Single_Cycle/MUX.v}
\subsubsection*{\large Module 2.11 - Program Counter}
\lstinputlisting[language = Verilog]{../RISC_V_Single_Cycle/Program_Counter.v}
\subsubsection*{\large Module 2.12 - Register File}
\lstinputlisting[language = Verilog]{../RISC_V_Single_Cycle/registerFile.v}
\subsubsection*{\large Module 2.13 - RISC-V Processor (Top Level Module)}
\lstinputlisting[language = Verilog]{../RISC_V_Single_Cycle/RISC_V_Processor.v}
\subsubsection*{\large Module 2.14 - Shift Left}
\lstinputlisting[language = Verilog]{../RISC_V_Single_Cycle/shift_left.v}
\subsubsection*{\large Module 2.15 - Test Bench}
\lstinputlisting[language = Verilog]{../RISC_V_Single_Cycle/tb.v}
\subsection*{\Large Section 3 - RISC-V Pipeline Processor Modules}
\subsubsection*{\large Module 3.1 - Adder}
\lstinputlisting[language = Verilog]{../RISCV_Pipeline/Adder.v}
\subsubsection*{\large Module 3.2 - ALU (64-bit)}
\lstinputlisting[language = Verilog]{../RISCV_Pipeline/ALU_64_bit.v}
\subsubsection*{\large Module 3.3 - ALU Control}
\lstinputlisting[language = Verilog]{../RISCV_Pipeline/ALU_Control.v}
\subsubsection*{\large Module 3.4 - Branch Module}
\lstinputlisting[language = Verilog]{../RISCV_Pipeline/branch_module.v}
\subsubsection*{\large Module 3.5 - Control Unit}
\lstinputlisting[language = Verilog]{../RISCV_Pipeline/Control_Unit.v}
\subsubsection*{\large Module 3.6 - Data Memory}
\lstinputlisting[language = Verilog]{../RISCV_Pipeline/Data_Memory.v}
\subsubsection*{\large Module 3.7 - EX/MEM Stage Register}
\label{ex_mem}
\lstinputlisting[language = Verilog]{../RISCV_Pipeline/EX_MEM.v}
\subsubsection*{\large Module 3.8 - Forwarding Unit}
\label{forward}
\lstinputlisting[language = Verilog]{../RISCV_Pipeline/forwarding_unit.v}
\subsubsection*{\large Module 3.9 - Hazard Detection Unit}
\label{hdu}
\lstinputlisting[language = Verilog]{../RISCV_Pipeline/Hazard_Detection.v}
\subsubsection*{\large Module 3.10 - ID/EX Stage Register}
\label{id_ex}
\lstinputlisting[language = Verilog]{../RISCV_Pipeline/ID_EX.v}
\subsubsection*{\large Module 3.11 - IF/ID Stage Register}
\label{if_id}
\lstinputlisting[language = Verilog]{../RISCV_Pipeline/IF_ID.v}
\subsubsection*{\large Module 3.12 - Immediate Data Generator}
\lstinputlisting[language = Verilog]{../RISCV_Pipeline/imm_data_gen.v}
\subsubsection*{\large Module 3.13 - Instruction Decoder}
\lstinputlisting[language = Verilog]{../RISCV_Pipeline/instruction.v}
\subsubsection*{\large Module 3.14 - Instruction Memory}
\lstinputlisting[language = Verilog]{../RISCV_Pipeline/Instruction_Memory.v}
\subsubsection*{\large Module 3.15 - MEM/WB Stage Register}
\label{mem_wb}
\lstinputlisting[language = Verilog]{../RISCV_Pipeline/MEM_WB.v}
\subsubsection*{\large Module 3.16 - MUX (64-bit, 2 by 1)}
\lstinputlisting[language = Verilog]{../RISCV_Pipeline/MUX.v}
\subsubsection*{\large Module 3.17 - MUX (64-bit, 3 by 1)}
\lstinputlisting[language = Verilog]{../RISCV_Pipeline/MUX_3.v}
\subsubsection*{\large Module 3.18 - MUX (ALU Control, 16 by 8)}
\label{muxc}
\lstinputlisting[language = Verilog]{../RISCV_Pipeline/MUX_Control.v}
\subsubsection*{\large Module 3.19 - Program Counter}
\label{pc}
\lstinputlisting[language = Verilog]{../RISCV_Pipeline/Program_Counter.v}
\subsubsection*{\large Module 3.20 - Register File}
\lstinputlisting[language = Verilog]{../RISCV_Pipeline/registerFile.v}
\subsubsection*{\large Module 3.21 - RISC-V Processor (Top Level Module)}
\label{tlmp}
\lstinputlisting[language = Verilog]{../RISCV_Pipeline/RISC_V_Pipeline.v}
\subsubsection*{\large Module 3.22 - Shift Left}
\lstinputlisting[language = Verilog]{../RISCV_Pipeline/shift_left.v}
\subsubsection*{\large Module 3.23 - Test Bench}
\lstinputlisting[language = Verilog]{../RISCV_Pipeline/tb.v}









\end{document}